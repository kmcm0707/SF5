\documentclass[a4paper, 11pt, oneside]{report}
\author{Kyle McMillan kcm40}
\title{SF5 Interim Report Two} 
\pagenumbering{roman}


\usepackage{geometry} % This package allows the editing of the page layout
\usepackage{amsmath}

\usepackage[utf8]{inputenc} % Required for inputting international characters
\usepackage[T1]{fontenc} % Output font encoding for international characters
\usepackage{fouriernc} % Use the New Century Schoolbook font
\usepackage{fancyhdr}
\usepackage{graphicx}
\usepackage{float}
\usepackage{multirow}
\usepackage{hyperref}
\usepackage{verbatimbox}
\setlength{\headheight}{14.49998pt}
\setlength{\textheight}{675pt}
\setlength{\textwidth}{456pt}
\addtolength{\topmargin}{-2.49998pt}


\pagestyle{fancy}
\fancyhf{}% Clear header/footer
\fancyhead[L]{3G1}
\fancyhead[R]{Kcm40}
\fancyfoot[C]{\thepage}% \fancyfoot[R]{\thepage}
\renewcommand{\headrulewidth}{0.4pt}% Default \headrulewidth is 0.4pt
\renewcommand{\footrulewidth}{0.4pt}
% custom commands

\newcommand{\centreImage}[2]{
	\begin{figure}[H]
		\centering
		\includegraphics[width=0.9\textwidth]{#1}
		\caption{#2}
        \label{fig:#1}
	\end{figure}
}
\newcommand{\centreImageTwo}[3]{
    \begin{figure}[H]
        \centering
        \includegraphics[width=0.33\textwidth]{#1}
        \includegraphics[width=0.33\textwidth]{#2}
        \caption{#3}
        \label{fig:#1}
    \end{figure}
}


\begin{document}

%----------------------------------------------------------------------------------------
%	TITLE PAGE
%----------------------------------------------------------------------------------------

\begin{titlepage}
    \centering % Centre everything on the title page
	
	\scshape % Use small caps for all text on the title page
	
	\vspace*{\baselineskip} % White space at the top of the page
    %------------------------------------------------
	%	Title
	%------------------------------------------------
	
	\rule{\textwidth}{1.6pt}\vspace*{-\baselineskip}\vspace*{2pt} % Thick horizontal rule
	\rule{\textwidth}{0.4pt} % Thin horizontal rule
	
	\vspace{0.75\baselineskip} % Whitespace above the title
	
	{\LARGE SF5 Interim Report Two} % Title
	
	\vspace{0.75\baselineskip} % Whitespace below the title
	
	\rule{\textwidth}{0.4pt}\vspace*{-\baselineskip}\vspace{3.2pt} % Thin horizontal rule
	\rule{\textwidth}{1.6pt} % Thick horizontal rule
	
	\vspace{2\baselineskip} % Whitespace after the title block
	
	%------------------------------------------------
	%	Subtitle
	%------------------------------------------------
	
	Part IIA - SF5: Networks, friendship, and disease % Subtitle or further description
	
	\vspace*{3\baselineskip} % Whitespace under the subtitle
	
	%------------------------------------------------
	%	Editor(s)
	%------------------------------------------------
	
	Edited By
	
	\vspace{0.5\baselineskip} % Whitespace before the editors
	
	{\scshape\Large Kyle McMillan \\ kcm40} % Editor list
	
	\vspace{0.5\baselineskip} % Whitespace below the editor list
	
    \textsc{University Of Cambridge} % Editor affiliation
	
	\vfill % Whitespace between editor names and publisher logo

\end{titlepage}

%----------------------------------------------------------------------------------------
\subsubsection*{Classes}
\begin{verbnobox}[\fontsize{8pt}{8pt}\selectfont \mbox{}]
class Network(object):
    def __init__(self , num_nodes):
        self.adj = {i:set() for i in range (num_nodes)}

    def add_edge(self , i , j):
        self.adj[i].add(j)
        self.adj[j].add(i)

    def neighbors(self , i):
        return self.adj[i]
    
    def edge_list(self):
        return [(i,j)for i in self.adj for j in self.adj[i] if i < j]
    
    def num_edges(self):
        return sum([len(self.adj[i]) for i in self.adj])//2
    
    def find_component(self, i):
        c = set()
        stack = [i]
        while stack:
            node = stack.pop()
            c.add(node)
            stack.extend(self.neighbors(node) - c)
        return c
    
    def degree_distributions_individual(self):
        return np.array([len(self.adj[i]) for i in self.adj])
    
    def degree_distributions(self):
        degrees = [len(self.adj[i]) for i in self.adj]
        return pd.Series(degrees).value_counts().sort_index()
    
    def friends_degree_distribution(self):
        return np.array([np.mean([len(self.adj[j]) for j in self.adj[i]]) for i in self.adj])

class configuration_graph(Network):
    def __init__(self , num_nodes , degree_sequence):
        super().__init__(num_nodes)
        S = np.array([ i for i in range (num_nodes) for _ in range (degree_sequence[i])])
        S = np.random.permutation(S)
        if len (S) % 2:
            S = S[:-1]
        S = S.reshape(-1 ,2)
        for i , j in S:
            self.add_edge(i , j)

class poisson_configuration_graph(Network):
    def __init__(self , num_nodes , lambda_):
        super().__init__(num_nodes)
        S = np.random.poisson(lambda_ , num_nodes)
        S = np.array([ i for i in range (num_nodes) for _ in range (S[i])])
        S = np.random.permutation(S)
        if len (S) % 2:
            S = S[:-1]
        S = S.reshape(-1 ,2)
        for i , j in S:
            self.add_edge(i , j)

class geometric_configuration_graph(Network):
    def __init__(self , num_nodes , p):
        super().__init__(num_nodes)
        S = np.random.geometric(p , num_nodes) - 1
        S = np.array([ i for i in range (num_nodes) for _ in range (S[i])])
        S = np.random.permutation(S)
        if len (S) % 2:
            S = S[:-1]
        S = S.reshape(-1 ,2)
        for i , j in S:
            self.add_edge(i , j)
 
class popular_poisson_configuration_graph(Network):
def __init__(self , num_nodes , lambda_, popular_nodes_num):
    super().__init__(num_nodes)
    S = np.random.poisson(lambda_ , num_nodes)
    S = np.array([ i for i in range (num_nodes) for _ in range (S[i])])
    S = np.random.permutation(S)
    if len (S) % 2:
        S = S[:-1]
    S = S.reshape(-1 ,2)
    for i , j in S:
        self.add_edge(i , j)
    popular_nodes = []
    temp = self.adj.copy()
    for i in range(popular_nodes_num):
        popular_nodes.append(max(temp, key=lambda x: len(temp[x])))
        temp.pop(max(temp, key=lambda x: len(temp[x])))
    for i in popular_nodes:
        for j in popular_nodes:
            if i != j:
                self.add_edge(i, j)

\end{verbnobox}
\subsubsection*{Images}
\centreImageTwo{poiss.png}{goem.png}{Degree Left: Poisson distribution, Right: Geometric distribution}
\centreImageTwo{Poisson_friend.png}{Geo_friend.png}{Friend Degree Left: Poisson distribution, Right: Geometric distribution}
\centreImageTwo{Poisson_delta.png}{Geo_delta.png}{Delta Graph Left: Poisson distribution, Right: Geometric distribution}
\centreImageTwo{Poisson_components.png}{Geometric_components}{Components Left: Poisson distribution, Right: Geometric distribution}
\centreImageTwo{Poisson_popular_components.png}{Popular_components.png}{Popular Components Left: Poisson distribution, Right: Geometric distribution}
\pagebreak
\subsubsection*{1) For a configuration model network sample the degree distributions created using a poisson and geometric distribution with mean 10 and 10000 nodes.}
Code:
\begin{verbnobox}[\fontsize{8pt}{8pt}\selectfont]
    degrees = [poisson_configuration_graph(10000, 10).degree_distributions()]
    # degrees = [geometric_configuration_graph(10000, 1/11).degree_distributions()]
    degrees = pd.DataFrame(degrees)
    combined_degrees = degrees
    values = combined_degrees.values.flatten()
    #print(values)
    coloumns = combined_degrees.columns
\end{verbnobox}

Figure 1 shows the degree distributions. This code generates a configuration model network with 10000 nodes and mean 10 degree distribution using a poisson and geometric distribution.
The degree distributions are then plotted in the figure above. The poisson distribution is on the left and the geometric distribution is on the right.
We can see that the means are roughly 10 for both distributions and both distributions follow the pmf.

\subsubsection*{2) Friend degree distribution}

Code:
\begin{verbnobox}[\fontsize{8pt}{8pt}\selectfont]
    friend_degrees_poisson = []
    for i in range(10000):
        neighbours = np.array(list(poisson_config_graph.neighbors(np.random.randint(0, 10000))))
        if len(neighbours) > 0:
            friend_degrees_poisson.append(len(poisson_config_graph.neighbors(np.random.choice(neighbours))))
            i-=1
\end{verbnobox}

Figure 2 shows the friend graph. From this can see the friend degree distribution is a lot larger for the geometric distribution than the poisson distribution but both are higher than the average degree distribution.

\subsubsection*{3) $\Delta_i = \kappa_i - k_i$}
$k_i$ - degree of node $i$
$\kappa_i$ - average degree of neighbours of node $i$
$$\kappa_i = \frac{1}{k_i} \sum_j A_{ij}k_j$$
In this histogram nodes of degree 0 are excluded as they would have a $\kappa_i$ of infinity using the above equation.
Code:
\begin{verbnobox}[\fontsize{8pt}{8pt}\selectfont]
    delta_friends = poisson_config_graph.friends_degree_distribution() - poisson_config_graph.degree_distributions_individual()
    zero_index = np.where(poisson_config_graph.degree_distributions_individual() == 0)
    delta_friends = np.delete(delta_friends, zero_index)
\end{verbnobox}

Figure 3 shows the delta graph. Results as expected. The $\Delta_i$ for the poisson distribution is centred around 1 and the geometric distribution is centred around 10. But both graphs similar to be a normal distribution.

\subsubsection*{4) Proving $q_k$ - the probability that a randomly chosen neighbour $k$.}

\begin{align}
    q_k &= P(k_j = k| k_i -> k_j) \\
    q_k &= \frac{\text{Number of nodes with degree k that are neighbours of a node with degree i}}{\text{All degrees}} \\
    q_k &= \frac{kn_k}{\sum_{k_2} k_2n_{k_2}} = \frac{k p_n n}{n \sum_{k_2} k_2 p_{k_2}} = \frac{k p_n}{\sum_{k_2} k_2 p_{k_2}}
\end{align}

\begin{align}
    E[\Delta_i] &= \frac{1}{\mu} \sum_{k} k^2 p_k - \mu = \frac{\sigma^2}{\mu} 
\end{align}

\subsubsection*{5) $q_k$ and generating function}
$$G'(z) = \sum^{\infty}_{k=1}kz^{k-1}p_k$$
\begin{align}
    Q(z) &= \frac{\sum_k z^k p_k k}{\sum_{k_2}k_2p_{k_2}} \\
    Q(z) &= \frac{zG'(z)}{G'(1)} 
\end{align}
Poisson distribution:

\begin{align}
    G(z) &= e^{\lambda(z-1)} \\
    G'(z) &= \lambda e^{\lambda(z-1)} \\
    Q(z) &= ze^{\lambda(z-1)} 
\end{align}
This is zP(z) so is $p_{k+1}$ - corresponds to Delta = 1 a lambda should be increased by 1.
Geometric distribution:

\begin{align}
    G(z) &= \frac{p}{1-(1-p)z} \\
    G'(z) &= \frac{p(1-p)}{(1-(1-p)z)^2} \\
    Q(z) &= \frac{zp^2}{(1-(1-p)z)^2} 
\end{align}
This is $(k+2)p^2(1-p)^{k+2}$ - which is geometric distribution squared and added 2 to k.

\subsubsection*{6) Components}
Code:
\begin{verbnobox}[\fontsize{8pt}{8pt}\selectfont]
    for i in range(11):
        num_components_one = [len(geometric_configuration_graph(10000, 1 - 2/3 * i/10).find_component(1)) for _ in range(40)]
        components.append(np.mean(num_components_one))
\end{verbnobox}

Figure 4 shows the components. The poisson graph starts increasing at 1 and the geometric graph starts increasing at 0.5.
This makes sense as expected number of neighbours for the poisson distribution is lambda and so as long as a node has one edge it's component will increase as friends more popular.
Same argument for geometric distribution but as the expected number of edges is $\frac{1-p}{p}$ this is only over 1 for p = 0.5.

\subsubsection*{7) Popular People}
For this it is likely all popular people are in the same component. Then other people will be effectively be by themselves, with a few unpopular people or effectively small
offshoots connected to the popular people. For both distributions have modelled it so people a certain number of the most popular people are connected together (increases their degree by 1 but ignorable).
The starting point where component size increases is the same as the previous question but the rate of increase is much higher.
\end{document}


